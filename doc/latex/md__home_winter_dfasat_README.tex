D\+F\+A\+S\+AT in C++

\subsubsection*{What is this repository for?}

\subsubsection*{How do I get set up?}

It needs the G\+NU scientific library (development) package. In Ubuntu, install via \$ apt-\/get install libgsl-\/dev

Have libpopt installed and run make clean all. If you want to use the S\+A\+T-\/solving part of it, get lingeling from \href{http://fmv.jku.at/lingeling/}{\tt http\+://fmv.\+jku.\+at/lingeling/} and run its build.\+sh, or a similar solver.

\subsubsection*{How do I run it?}

Run ./dfasat --help to get help.

Example\+:

\$ ./dfasat -\/h alergia -\/d alergia\+\_\+data -\/n 1 -\/A 0 -\/D 2000 inputfile \char`\"{}/bin/true\char`\"{}

h defines the algorithm, and d the data type. n defines the number of iterations, which is useful if you use the S\+A\+T-\/solver. -\/A and -\/D are used to decide when to switch from the heuristic to the S\+A\+T-\/solver. The parameters in the example essentially prevent the use of the solver.

The thresholds for relevance of states and symbols can be adjusted using -\/q and -\/y.

\subsubsection*{Output files}

D\+F\+A\+S\+AT will generate several files\+:


\begin{DoxyItemize}
\item pre\+\_\+dfa1.\+dot, is a dot file of the problem instance send to the S\+AT solver
\item dfa1.\+dot is the result in dot format if you provided a S\+A\+T-\/solver
\end{DoxyItemize}

You can plot the dot files via

\begin{quote}
\$ dot -\/\+Tpdf file.\+dot -\/o outfile.\+pdf \end{quote}


after installing dot from graphviz.

\subsubsection*{Contribution guidelines}


\begin{DoxyItemize}
\item Fork and implement, request pulls.
\item You can find sample evaluation files in ./evaluation. Make sure to R\+E\+G\+I\+S\+T\+ER your own file to be able to access it via the -\/h flag.
\end{DoxyItemize}

\subsubsection*{Who do I talk to?}


\begin{DoxyItemize}
\item Sicco Verwer
\item one of his PhD students\+: Qin, Nino, or Chris 
\end{DoxyItemize}